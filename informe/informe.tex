\documentclass[a4paper,11pt]{article}

\usepackage[utf8]{inputenc}
\usepackage[spanish]{babel}
\usepackage{amsmath}
\usepackage{graphicx}
\usepackage{hyperref}
\usepackage{fancyhdr}
\usepackage{geometry}

\geometry{margin=1in}
\pagestyle{fancy}
\fancyhf{}
\fancyhead[L]{Práctica de Programación}
\fancyhead[R]{Martínez \& Vallejos}
\fancyfoot[C]{\thepage}

\title{Informe de Práctica: Implementación del 3-en-Raya en python}
\author{Adrián Martínez Pérez \\ Guillem Arnau Vallejos}
\date{Enero 2025}

\begin{document}

\maketitle
\tableofcontents
\newpage

\section{Introducción}
La presente práctica tiene como objetivo la implementación de un "tablero abstracto" que permita jugar al clásico juego del 3-en-raya, así como a algunas de sus variantes. El desarrollo de este proyecto busca fomentar el trabajo en equipo, la comprensión de conceptos de programación funcional y el diseño de estructuras de datos reutilizables.

El informe detalla el proceso de diseño, implementación, pruebas y validación del sistema, así como las decisiones tomadas durante su desarrollo. Finalmente, se incluye una reflexión sobre las dificultades enfrentadas y los aprendizajes adquiridos.

\section{Descripción del Juego}
El 3-en-raya es un juego de estrategia en el que dos jugadores colocan sus piezas en un tablero de $3 \times 3$ casillas. El objetivo en la variante estándar es alinear tres piezas consecutivas en horizontal, vertical o diagonal.

En esta práctica, se consideran las siguientes variantes del juego:
\begin{itemize}
    \item \textbf{Juego estándar}: El primer jugador en alinear tres piezas gana.
    \item \textbf{Variante "misery"}: El primer jugador que alinea tres piezas pierde.
\end{itemize}

El tablero se ha diseñado para ser flexible, permitiendo ajustar su tamaño y el número de piezas por jugador.

\section{Diseño e Implementación}
El programa desarrollado consta de dos componentes principales:
\begin{itemize}
    \item \textbf{El tablero abstracto}: Implementado en el archivo \texttt{abs\_board.py}, contiene las estructuras de datos y funciones necesarias para gestionar el estado del juego.
    \item \textbf{Drivers}: Programas que permiten la interacción con los jugadores. Se han utilizado tanto interfaces basadas en texto como en gráficos (usando Pygame).
\end{itemize}

El diseño del tablero abstracto se basa en principios de programación funcional, empleando clausuras y variables \texttt{nonlocal} para encapsular el estado del juego. Además, se ha asegurado que el código sea modular y fácil de extender.

\section{Pruebas y Validación}
Para garantizar el correcto funcionamiento del sistema, se realizaron las siguientes pruebas:
\begin{itemize}
    \item Simulaciones de partidas completas en ambas variantes del juego.
    \item Validación cruzada con los drivers proporcionados y modificados.
\end{itemize}

Los resultados obtenidos coinciden con las expectativas y confirman que el sistema implementado cumple con los requisitos de la práctica.

\section{Dificultades y Aprendizajes}
Durante el desarrollo, enfrentamos las siguientes dificultades:
\begin{itemize}
    \item Comprensión y aplicación de clausuras en Python.
    \item Diseño modular que permita cambios en las reglas del juego sin afectar el núcleo del sistema.
\end{itemize}

Como resultado, hemos adquirido habilidades avanzadas en programación funcional y diseño de software colaborativo.

\section{Reparto de Responsabilidades}
El trabajo se distribuyó de la siguiente manera:
\begin{itemize}
    \item Adrián Martínez Pérez: Implementación del tablero abstracto y pruebas con drivers.
    \item Guillem Arnau Vallejos: Diseño de variantes del juego y desarrollo de interfaces gráficas.
\end{itemize}

\section{Conclusión}
El desarrollo de esta práctica ha sido una experiencia enriquecedora que nos ha permitido consolidar conceptos clave de programación y trabajo en equipo. El sistema implementado cumple con los objetivos planteados y ofrece una base sólida para futuras mejoras.

\end{document}
