\documentclass[a4paper,12pt]{article}
\usepackage[utf8]{inputenc}
\usepackage[spanish]{babel}
\usepackage{graphicx}
\usepackage{geometry}
\usepackage{hyperref}
\geometry{margin=1in}

\title{Informe de Práctica: Implementación del Juego 3 en Raya}
\author{Adrián Martínez Pérez \\ Guillem Arnau Vallejos}
\date{Facultad de Informática de Barcelona\\Grado en Inteligencia Artificial}

\begin{document}

\maketitle

\section*{Introducción}
El objetivo de esta práctica es implementar un tablero abstracto para el juego 3 en raya, utilizando el módulo \texttt{pygame} de \texttt{python3}. Inicialmente, en esta primera versión, hemos desarrollado el juego para ser jugado por dos personas, y no contra una máquina. El juego consta de cuatro archivos principales:

\begin{itemize}
    \item \textbf{\texttt{abs\_board\_h.py}}: Funciones y lógica necesarias para gestionar el estado del tablero, verificar condiciones de victoria y manejar los movimientos de los jugadores.
    \item \textbf{\texttt{utils.py}}: Implementa las utilidades del juego: tanto las pantallas de carga en el modo texto (\texttt{txt}) como en la interfaz gráfica (\texttt{gui}), y el menú principal.
    \item \textbf{\texttt{constants.py}}: Este archivo sirve para almacenar todas las constantes necesarias para el funcionamiento del programa.
    \item \textbf{\texttt{main\_gui.py}}: Implementa la interfaz gráfica de usuario. Contiene la configuración inicial, creación de ventanas e integración con \texttt{abs\_board\_h.py}.
    \item \textbf{\texttt{main\_txt.py}}: Implementa la versión para consola del juego, con colores y mejoras visuales para un mayor atractivo.
\end{itemize}

En la carpeta de la práctica se incluye todo lo necesario para la ejecución y disfrute del juego. Solo es necesario ejecutar dos archivos para jugar al juego:

\begin{itemize}
    \item Para jugar en modo texto, ejecutar en la consola el siguiente comando:
    \begin{verbatim}
    $ python3 main_txt.py
    \end{verbatim}
    \item Para jugar en modo gráfico, ejecutar en la consola el siguiente comando:
    \begin{verbatim}
    $ python3 main_gui.py
    \end{verbatim}
\end{itemize}

\section*{Metodología de la Implementación}
Para la implementación de este código hemos trabajado dos personas: Adrián Martínez Pérez y Guillem Arnau Vallejos, mediante un repositorio compartido en la plataforma \texttt{GitHub}, para poder gestionar el historial de versiones del juego en caso de que cualquier cosa falle. GitHub ha sido una excelente plataforma para la colaboración en este tipo de proyectos, y hemos disfrutado mucho aprendiendo a utilizarla.

No hemos tenido un reparto estricto de las tareas, sino que íbamos programando y comunicándonos constantemente sobre todo lo que salía bien, mal, futuras implementaciones y posibles mejoras.

\end{document}
